\documentclass[10pt,a4paper]{article}

\usepackage[margin=0.7in]{geometry}
\usepackage{amssymb, amsthm, amsmath, amsfonts}
\usepackage{array, xcolor, enumitem, graphicx, scrextend}
\usepackage{cancel, booktabs}

\usepackage[czech]{babel}
\usepackage[utf8]{inputenc}
\usepackage[unicode]{hyperref}
\usepackage[useregional]{datetime2}

\hypersetup{
    colorlinks=true,
    linkcolor=black,
    urlcolor=blue,
    pdftitle={Algebra - zpracované příklady ke zkoušce},
}

% tahak - prikazy %
% \includegraphics{obrazek.png}         % import konkretniho obrazku
% a~b                                   % mezera mezi pismeny 'a' a 'b'
% a \quad b                             % velka mezera mezi pismeny 'a' a 'b'
% \sim                                  % ~
% \Longleftarrow                        % <==


% zkratky %
\newtheorem{veta}{Věta}
\newtheorem{definice}{Definice}
\newtheorem{tvrzeni}{Tvrzení}
\newtheorem{lemma}{Lemma}
\newtheorem{pozorovani}{Pozorování}
\newtheorem{dusledek}{Důsledek}
\newtheorem{priklad}{Příklad}
\newtheorem{poznamka}{Poznámka}
\newtheorem{fakt}{Fakt}
\newtheorem{algoritmus}{Algoritmus}

\newcommand{\N}{{\mathbb{N}}}       % prirozena cisla
\newcommand{\Z}{{\mathbb{Z}}}       % cela cisla
\newcommand{\Zs}{{\mathbb{Z}_n^*}}  % cela cisla vcetne nekonecna
\newcommand{\Q}{{\mathbb{Q}}}       % racionalni cisla
\newcommand{\R}{{\mathbb{R}}}       % realna cisla
\newcommand{\Cc}{{\mathbb{C}}}      % komplexni cisla
\newcommand{\F}{{\mathbb{F}}}       % teleso
\newcommand{\Pp}{{\mathcal{P}}}     % potencni mnozina
\newcommand{\RR}{{\mathcal{R}}}      % okruh/ring
\newcommand{\ds}{{\displaystyle}}   % displaystyle

\DeclareMathOperator{\lcm}{lcm}

\newcommand{\hr}{{\begin{center}\par\rule{\textwidth}{0.5pt} \end{center}}}
\newcommand\makesmall{\fontsize{8pt}{11pt}\selectfont}

\renewcommand*{\contentsname}{Obsah}
\renewcommand*{\proofname}{Důkaz:}
\renewcommand*{\figurename}{\makesmall Obr.}

% hlavicka
\setlength{\parindent}{0em}

\title{Algebra - zpracované příklady ke zkoušce}
\date{\today}
\author{\sc Karel Velička}

\graphicspath{ {img/} }             % obrazky ulozeny ve slozce img/
\newcommand*{\threeemdash}{\rule[0.5ex]{6em}{0.55pt}}

\begin{document}
\pagenumbering{arabic}
\maketitle
% podnadpis + obsah 
\begin{center}
    Doc. Mgr. et Mgr. Jan Žemlička Ph.D.
\end{center}

\tableofcontents
\newpage

% obsah dokumentu


%%%%%%%%%%%%%%%%%%%%%%%%%%%%%%%%%%%%%%%%%%%%%%%%%%%
%%% TEORIE CISEL %%%%
%%%%%%%%%%%%%%%%%%%%%

\section{Teorie čísel}

\subsection{Modulární aritmetika}

\subsubsection{Najděte $u, v \in \Z$, pro která $103u + 77v = 1$.}

Hledáme Bézoutovy koeficienty pro $\gcd(77, 103) = 1$ podle vzorce $(u_{i+1}, v_{i+1})=(u_{i-1}, v_{i-1})+q_i(u_{i}, v_{i})$.

Pro $i \in \{0, \dots, 4\}: ~a_i = (103, 77, 26, 25, 1), \quad u_i=(1, 0, 1, -2, 3), \quad v_i = (0, 1, -1, 3, -4)$, takže: $$103 \cdot 3 - 77 \cdot 4 = 1.$$

\subsubsection{Vypočítejte poslední číslici $33^{999}$.}

Počítáme poslední cifru, tedy $33^{999} \mod 10$. Použijeme Eulerovu větu: $\varphi(10) = 4$ a $\gcd(33, 10)=1$ platí. A také využijeme faktu, že $33 \mod 10 = 3$.

\[
    33^{999} \equiv 33^{1000} \cdot 3^{-1} \equiv \underbrace{33^{4\cdot 250}}_{(33^4)^{250}=1^{250}} \cdot ~ 3^{-1} = 1 \cdot 3^{-1} \equiv 7
\]


\subsubsection{Vypočítejte $2023^{2022^{2021}} \mod 101$.}


Nejprve zmodulíme $2023 \pmod {101} = 3$ a dosadíme do původní rovnice: $3^{2022^{2021}} \pmod {101}$.

Dále si můžeme uvědomit, díky Eulerově větě, že: $3^{2022^{2021} \mod n} \pmod {101}$, kde $n = \varphi(101) = 100$ (protože $101$ je prvočíslo).
Tedy spočítáme $a:= 2022^{2021} \pmod {100}$ a dosadíme do $3^a \pmod {101}$.

\begin{flalign*}
    2022^{2021} \pmod{100} &\equiv 22^{2021} \pmod{100} \textit{, spočítáme Eulerovu funkci pro } \varphi(100):\\
    \varphi(100) = \varphi(4 \cdot 25) &= 2\cdot 4 \cdot 5 = 40 \textit{, dostaneme tak: } 22^{40} \equiv 1 \pmod {40} \textit{ a upravíme původní výraz:}\\
    (22^{40})^{50} \cdot 22^{21} \pmod {100} &\equiv 1^{50} \cdot 22^{21} \pmod {100} = 22^{21} \pmod {100}
\end{flalign*}

Nyní tedy hledáme řešení pro $22^{21} \pmod {100}$, a protože $100$ můžeme rozepsat jako $100 = 25 \cdot 4 \implies \gcd(25,4)=1$, můžeme k tomu použít Čínskou zbytkovou větu:

\begin{flalign*}
    x &= 22^{21} \pmod {25} = 22\\
    x &= 22^{21} \pmod 4 = 22^{20} \cdot 22^1 \pmod 4 \equiv 0\cdot 22 \pmod 4 = 0
\end{flalign*}

Víme, že výsledek by měl být ve formátu $22^{21} \pmod {100} = (a_1 \cdot b_1 \cdot m_1) + (a_2 \cdot b_2 \cdot m_2) \mod 100$, kde už tedy máme $a_1 = 22$ a $a_2 = 0$. Ostatní členy s indexem $1$ dopočítáme, s indexem $2$ nemusíme díky $a_2=0$:

$b_1 = 4 \quad$ a $ \quad m_1 = b_1^{-1} \pmod {25} = 4^{-1} \pmod {25} = 19$

A dostáváme tak: $$22^{21} \pmod {100} = 22 \cdot 4 \cdot 19 \pmod{100} = 1672 \pmod {100} \equiv 72$$

Nyní dosadíme do původní rovnice $3^{72} \pmod{101}$ a podle tabulky už určíme mocninu:

\begin{center}
    \begin{tabular}{ |c|c|c| } 
     \hline
     $3^n$ & úprava &  $\mod 101$ \\ \hline \hline
     $3^1$ & $3$ &  $3$ \\ 
     $3^2$ & $3\cdot 3$ &  $9$ \\ 
     $3^3$ & $3\cdot 9$ &  $27$ \\ \hline
     $3^6$ & $(3^3)^2 = 27^2 = 729$ &  $22$ \\
     \hline
    \end{tabular}
\end{center}

Dostáváme $3^{72} = (3^6)^{12} = 22^{12} \pmod{101}$ a opět podle tabulky:

\begin{center}
    \begin{tabular}{ |c|c|c| } 
     \hline
     $22^n$ & úprava &  $\mod 101$ \\ \hline \hline
     $22^1$ & $22$ &  $22$ \\ 
     $22^2$ & $22\cdot 22 = 484$ &  $80$ \\ 
     $22^3$ & $22\cdot 80 = 1760$ &  $43$ \\ 
     $22^4$ & $22\cdot 43 = 946$ &  $37$ \\ 
     $22^5$ & $22\cdot 37 = 814$ &  $6$ \\ 
     $22^6$ & $22\cdot 6 = 132$ &  $31$ \\ \hline
     $22^{12}$ & $(22^6)^2 = 31^2 = 961$ &  $52$ \\
     \hline
    \end{tabular}
\end{center}

Takže $2023^{2022^{2021}} \pmod{101} \equiv 3^{72} \equiv 22^{12} \equiv 52$.

\subsubsection*{Vyřešte systém kongruencí:}
\begin{flalign*}
    x &\equiv 1 \pmod 3\\
    x &\equiv 2 \pmod 5  ~~ \leadsto ~~ x = 5k + 2 \\
    x &\equiv 1 \pmod 7\\
    -----&------------\\
    5k + 2 &\equiv 1 \pmod 7, \quad (k\in \Z)\\
    5k &\equiv -1 \equiv 6 \pmod 7\\
    k &= 4 \implies\\
    \implies x &= 5\cdot 4 + 2 = 22 \in \Z_{35}\\
    -----&-------------\\
    22 + 35l &\equiv 1 \pmod 3, \quad (l\in \Z)\\
    35 l &\equiv -21 \equiv 0 \pmod 3\\
    35 l &\equiv 0 \pmod 3\\
    l &= 0 \implies \\
    \implies x &= 22 + 35 \cdot 0 = 22 \in \Z_{105}
\end{flalign*}

Celkově je řešením množina $\{x = 22 + 35 l \mid x < 105, l \in \Z\}$, tedy $x \in \{22, 57, 92\}$.

\newpage
%%%%%%%%%%%%%%%%%%%%%%%%%%%%%%%%%%%%%%%%%%%%%%%%%%%
%%%% POLYNOMY %%%%
%%%%%%%%%%%%%%%%%%


\section{Polynomy}

\subsection{Tělesa, okruhy, obory}

\subsubsection{Vypočítejte $33^{-1}$ v tělese $(\Z_{37}, +, \cdot -, 0)$.}

Máme těleso $\Z_{37}$ a hceme vypočítat inverz $33^{-1}$.
Ověříme nejprve, že $\gcd(33, 37)=1$. 
Dále hledáme Bézoutovy koeficienty $u$ a $v$ takové, že  $33u + 37v = 1$:

\begin{flalign*}
    \text{Pro } i\in \{0, \dots 3\}: ~a_i = (37, 33, 4, 1)&, \quad u_i=(1, 0, 1, 8), \quad v_i=(0, 1, 1, 9)\\
    \implies &33\cdot9 - 37\cdot8=1
\end{flalign*}

Takže $33^{-1}$ v tělese $\Z_{37}$ je $9$.

\subsubsection{Zkostruujte těleso o $125$ prvcích.}

Potřebujeme najít prvočíslo $p$ a $k\in \N$ tak, aby platil vztah $p^k=125$.

Jednou možností je vzít $p = 5$ a $k = 3$, protože $5^3 = 125$. 
Tímto způsobem získáme těleso se $125$ prvky.

\subsection{Dělitelnost, UFD}

\subsubsection{Dokažte, že $4x^3 - 15x^2 + 60x + 180$ je ireducibilní v $\Q[x]$ (Eisensteinovo kritérium)}

Pokud $\exists p \in Q$ ireducibilní prvek splňující $p \mid a_0$, $p | a_1 ,\dots , p | a_{n-1}$ a $p^2 \nmid a_0$ , pak je polynom ireducibilní v $\Q[x]$. 

Hledáme proto $p$, které dělí první až předposlední koeficient. V našem případě je to $p=5$.

Platí, že $5$ je ireducibilní, zároveň platí $5\mid 180, ~5\mid 60, ~5\mid 15$ a dokonce i $5^2 \nmid 180 \equiv 25 \nmid 180$.

Eisensteinovým kritériem jsme určili, že polynom $4x^3 - 15x^2 + 60x + 180$ je ireducibilní.


\subsubsection{V $\Z_2[x]$ najděte všechny ireducibilní polynomy stupně nejvýše $4$.}

Budeme postupně zkoušet všechny možnosti s tím, že předem nějaké vyloučíme. 

Například vyloučíme možnosti bez $x^0$ - například: $x^2+x=x(x+1)$, protože pak bychom mohli vytknout $x$.

Ze stejného důvodu vyřadíme všechny $x^i$ a $x^i+1$ pro $i={2,3,4}$. Obecně řečeno odstraníme všechny polynomy, které jsou nějakým násobkem našeho polynomu.
\begin{itemize}
    \item [$\deg 0$:] Nic.
    \item [$\deg 1$:] Pouze $x$ a $x+1$ jsou ireducibilní.
    \item [$\deg 2$:] Pouze $x^2+x+1$ je ireducibilní.
    \item [$\deg 3$:] Pouze $x^3+x^2+1$ a $x^3+x+1$ jsou ireducibilní.
    \item [$\deg 4$:] Pouze $x^4+x^3+x^2+x+1, ~~x^4+x^3+1,~~ x^4+x+1$
\end{itemize}

\subsubsection{Napiště $2x^2 - 6$ jako násobek ireducibilních polynomů v (a) $\Z[x]$, (b) $\Q[x]$, (c) $\Cc[x]$}

\begin{enumerate}[label=(\alph*)]
    \item $\Z[x]:$ Vytkneme $2$ a dostaneme výsledný polynom $f(x)=2(x^2-3)$, musíme ověřit ireducibilu.

    Číslo $2$ je ireducibilní, protože je prvočíslo. 
    Polynom $(x^2-3)$ je také ireducibilní, protože může být rozloženo pouze na $(x + \sqrt3)(x- \sqrt3)$, což ovšem nemá celočíselný kořen $\sqrt3 \notin \Z[x]$.
    \item $\Q[x]:$ $2$ jako polynom stupně nula nad tělesem je invertibilní, jinak také $\sqrt{3} \notin \Q[x]$, takže $2x^2-6$
    \item $\Cc[x]:$ Podobně jako v $\Q[x]$, akorát $\sqrt{3}\in \Cc[x] \implies f(x)=(2x + 2\sqrt 3)(x -\sqrt3)$. 
        
    Stačí určit ireducibilitu $(x \pm \sqrt{3})$. Jsou ireducibilní, protože jsou stupně $1$. 
\end{enumerate}    

\subsection{GCD a Modulo polynom}
\subsubsection{Show that $m(\alpha) = \alpha^3 + \alpha + 1$ is irreducible in the domain $\Z_7 [\alpha]$. Solve the equation $(\alpha^2 + 3)x + \alpha + 4 = \alpha^2$ in the feld $\Z_7 [\alpha]/(m(\alpha))$.}
\subsubsection{Vypočítejte $\gcd(x^5 + x^2 + x + 1, ~~x^3 + x + 1 \in \Z_2[x])$ a určete Bézoutovy koeficienty.}
\subsubsection{Vypočítejte $\gcd(5 - 3i, 7 + i)$ v oboru $\Z[i]$}

Budeme řešit za pomoci Eukleidova algoritmu v oboru $\Z[i] = \{a+bi \mid a,b\in \Z\}$.

$ $

Abychom věděli, co čím dělit, musíme určit normy: $\mathcal{V}(a)=||7+i|| = 50 > 34 = ||5-3i|| = \mathcal{V}(b)$. Dělíme:

\[
    \frac{7+i}{5-3i} = \frac{(7+i)(5+3i)}{(5-3i)(5+3i)} = \frac{35 + 21i + 5i-3}{25+9} = \frac{16}{17} + \frac{13}{17}i
\]

Nyní zvolím vhodné $q,r$ takové, aby $\mathcal{V}(r) < \mathcal{V}(5-3i)$. Protože $\frac{16}{17} + \frac{13}{17}i = \frac{1}{17}(16 + 13i) = \frac{13}{17}(\approx1 + i)$, zvolíme jako $q = 1+i$ a dopočítáme $r$.
\[
    (7+i) - (5-3i)(1+i) = 7+i - 5 -5i + 3i -3 = -1 -i = r
\]
Vidíme, že zjevně platí $\mathcal{V}(-1-i) = ||-1-i||= 2 < 34 =\mathcal{V}(5-3i)$. Budeme proto pokračovat v algoritmu:

\[
    \frac{5-3i}{-1-i} = \frac{-5+3i}{1+i} = \frac{(-5+3i)(1-i)}{(1-i)(1+i)} = \frac{-5+5i+3i+3}{2} = \frac{-2+8i}{2} = -1+4i
\]
Vidíme, že zbytek po dělení je $0$, proto $\gcd(5 - 3i, 7 + i)=-1-i$.

%%%%%%%%%%%%%%%%%%%%%%%%%%%%%%%%%%%%%%%%%%%%%%%%%%

\subsection{Aplikace}

\subsubsection{Reed-Solomonovy kódy}
Mějme těleso $T:=\F_8=\Z_2[\alpha]/(\alpha^3+\alpha + 1)$ a Reed-Solomonův $(2,4)-$kód nad abecedou $T$ pro $$u_1=1, ~~u_2=\alpha, ~~u_3 = \alpha^2 \text{ ~a~ } u_4 = \alpha + 1.$$

Tento kód může opravit jednu chybu.
\begin{enumerate}[label=(\alph*)]
    \item Zakóduj $(0,\alpha)$.
    \item Obdrželi jsme kód $(\alpha,\alpha^2, \alpha + 1,\alpha^2)$. Co bylo původní slovo?
    \item Obdrželi jsme slovo $w = (0,0, 1,1)$, ale kanál byl nespolehlivý.\\
    Ukažte, že toto slovo nelze dekódovat. Explicitně se po vás chce, abyste:\\
    - ukázali, že neexistuje kód $c$ s Hammingovou vzdáleností $\delta(c,w) \leq 1$.\\
    - nalezli dva kódy $c_1 ,c_2$ takové, že $\delta(c_1 ,w) = \delta(c_2 ,w) = 2$.
\end{enumerate}

\hr

\begin{enumerate}[label=(\alph*)]
    \item Zprávu $(0, \alpha)$ rozepíšeme jako polynom $f(x)=\sum \alpha_ix^i \implies f(x)=0 x^0 + \alpha x^1 \implies f(x)=\alpha x$.
    
    A dopočítáme kód $(f(u_1), f(u_2), f(u_3), f(u_4))$:
    \begin{itemize}
        \item $f(u_1) = \alpha \cdot u_1 = 1\cdot \alpha = \alpha$
        \item $f(u_2) = \alpha \cdot u_2 = \alpha \cdot \alpha = \alpha^2$
        \item $f(u_3) = \alpha \cdot u_3 = \alpha^2 \cdot \alpha = \alpha^3 = -\alpha - 1 = \alpha+1$
        \item $f(u_4) = \alpha \cdot u_4 = (\alpha+1)\cdot \alpha = \alpha^2 + \alpha$
    \end{itemize}
    Takže po zakódování dostaneme výsledný kód: $(\alpha, \alpha^2, \alpha+1, \alpha^2+\alpha)$.
    
    \item Máme kód $c=(c_1, c_2, c_3, c_4)=(\alpha,\alpha^2, \alpha + 1,\alpha^2)$ s maximálně jednou chybou a víme, že $c_i = f(u_i)$, pro $i\in\{1, 2, 3, 4\}$. 
    Potřebujeme určit polynom, z kterého odvodíme původní zprávu, což je zjevně opět $f(x)=\alpha x$.
        
    Naše obdržená zpráva splňuje podmínku jedné chyby se zprávou z $(a)$, protože nesedí pouze $c_4$.
    
    Původní slovo tak je opět $(0, \alpha).$

    \item Slovo nejde dekódovat, protože Hammingova vzdálenost pro $w$ je nejméně $2$.
    Víme, že lze opravit nejvýše $\lfloor \frac{d-1}2 \rfloor$ chyb $\implies d\geq n-k+1 \implies d\geq 4-2+1 \implies d\geq 3$.

    Celkem proto platí, že můžeme opravit nejvýše $\lfloor \frac{3-1}2 \rfloor = 1$ chybu, takže $3$ výstupy musí být správné:

    \begin{itemize}
        \item Buď dvě $0$ a jedna $1 \implies f(x)=0 \implies (0,0,0,0)$, takže $\delta(c,w) > 1$, protože $u_3, u_4$ neodpovídá
        \item Nebo dvě $1$ a jedna $0 \implies f(x)=1 \implies (1,1,1,1)$, takže $\delta(c,w) > 1$, protože $u_1, u_2$ neodpovídá
    \end{itemize}

    Slovo proto nelze dekódovat, protože v něm máme více než $2$ chyby, tedy $\delta(c ,w) \geq 2$

    Dva kódy $c_1, c_2$ s $\delta(c_1 ,w) = \delta(c_2 ,w) = 2$:
    \begin{itemize}
        \item $(0, 0) \leadsto f(x)=0 \leadsto c_1=(0,0,0,0)$ a tedy $\delta((0,0,0,0), (0,0,1,1)) = 2$.
        \item $(1,0) \leadsto f(x) = 1 \leadsto c_2 = (1,1,1,1)$ a tedy $\delta((1,1,1,1), (0,0,1,1)) = 2$.
    \end{itemize}
\end{enumerate}
\subsubsection{Sdílení klíčů}
Navrhněte schéma sdílení tajemství pro sedm účastníků - 
dva králové a pět eforů tak, že tajemství mohou rekonstruovat buď oba králové, nebo jeden král a všech pět eforů.
\begin{enumerate}[label=(\alph*)]
    \item Tajemstvím je konkrétní prvek tělesa $T$. Volba tělesa je na vás a výběr tajemství je na vás.
    \item Pravděpodobnost, že někdo náhodně uhodne tajemství, je menší než $2\%$.
\end{enumerate}
\hr

\begin{enumerate}
    \item Inspiroval jsem se učebnicovým příkladem, konkrétně Shamirovým protokolem a zvolil jsem těleso $T=\Z_{2^m}$ s pravděpodobností $\frac{1}{|T|}=(\frac{1}2)^m$.
    Aby pravděpodobnost byla $<2\%$, musíme volit $m\geq6$. Já jsem se rozhodl pro $\Z_{2^{256}}$. Tajemství je schováno v absolutním členu polynomu, $t=f(0)$.
    \item Počet klíčů pro $2$ krále musí být stejný jako počet klíčů pro $1$ krále a $5$ eforů. 
    $$2k = k+5 \implies k=5$$
    Což by znamenalo $5$ klíčů pro krále a $1$ klíč pro každého z $5$ eforů, tedy celkem $10/15$ klíčů pro odhalení tajemství.
    Vytváříme proto $(10,15)-$schéma a volíme tak polynom stupně $<10$: 
    $$f(x) = t + a_1x + a_2x^2 + a_3x^3+a_4x^4+a_5x^5 + a_6 x^5 + a_7 x^7 + a_8x^8 + a_9x^9$$
    
    \item Vygenerujeme $15$ náhodných hodnot $\alpha_1 \dots \alpha_{15} \in \Z_{2^8}$ a ty rozdáme po $5$ králům a po $1$ eforům.
    \begin{itemize}
        \item První král dostane vygenerované klíče $f(\alpha_i)$, kde $i \in \{1, \dots, 5\}$
        \item Druhý král dostane vygenerované klíče $f(\alpha_i)$, kde $i \in \{6, \dots, 10\}$
        \item Každý efor $e_j$ dostane vygenerované klíče: $f(\alpha_i)$, kde $j \in \{1, \dots, 5\}$ a $i = 10+j$
    \end{itemize}
    Jakmile se sejdou $2$ králové, nebo $1$ král a $5$ eforů, interpolují dokud jim nevyjde polynom $<10$. Ve chvíli kdy se tak stane, vezmou absolutní člen, což je tajemství.

    Pokud se sejdou v jiném počtu a dají dohromady $<10$ klíčů, vyjde jim polynomů stupně $<10$ mnoho a pravděpodobnost, že klíč uhodnou bude $\frac{1}{2^8} = 0.4\%$.
\end{enumerate}

\subsubsection{RSA}
Mějme systém RSA s veřejným klíčem $(N, e)=(91, ~5)$.
\begin{enumerate}[label=(\alph*)]
    \item Zašifrujte zprávu $x=4$ za pomoci klíče $(91, ~5)$.
    \item Protože jsme si vybrali malé $N$, je možné dešifrovat zprávu bez veřejného klíče. Dešifrujte zprávu $y=61$. Co bylo původní zprávou?
    \item Mějme jiný veřejný klíč $(N, e) = (169, ~5)$. Najděte $d$ a číslo $0<x<169$ takové, že po dešifrování veřejným klíčem $(169, ~5)$ vrátí RSA hodnotu různou od $x$.
\end{enumerate}

\hr

\begin{enumerate}[label=(\alph*)]
    \item Zašifrování probíhá způsobem $y = (x^e) \mod N$. V našem případě pro $x=4, ~ e=5, ~ N=91$:
    \begin{flalign*}
        y &= (4^5) \mod 91\\
        y &= 1024 \pmod {91}\\
        y &= 23 \pmod {91}
    \end{flalign*}
    \item Pokud chceme zprávu dešifrovat, musíme použít $x = y^d \pmod {N}$, kde $d$ je tajný klíč. 
    
    Ten sice neznáme, můžeme ho ale získat vztahem $d e \equiv 1 \pmod{\varphi (N)}$, kde $\varphi(N) = (p-1)(q-1)$.

    Máme $N = 91$, tedy jediná varianta pro prvočísla jsou $p=7, q=13$ (a naopak). Proto $\varphi(N)=12\cdot 6=72$

    $ $

    Dosadíme do $d e \equiv 1 \pmod{\varphi (N)}$ a dostáváme $5d \equiv 1 \pmod{72}$. A protože $\gcd(5,72)=1$, můžeme $d$ určit za pomoci euklidova algoritmu:
    \begin{flalign*}
        5d &\equiv 1 \pmod{72}  \quad // \quad 5\cdot 29=145 \equiv 1\\
        145d &\equiv 29 \pmod{72}\\
        d &\equiv 29 \pmod{72} \implies d = 29 + 72k ~(\forall k \in \Z)
    \end{flalign*}
    Stačí nám už jen dopočítat $x$, to uděláme za pomoci $x = (y^d) \mod N$:
    \begin{flalign*}
        x &= (y^d) \mod N \quad // \quad y=61, ~ N = 91, ~ d =29\\
        x &= (61^{29}) \mod 91\\
        x &= 3
    \end{flalign*}
    \item Máme zadáno $N = 169$ a $e=5$. A lehce si odvodíme $p = q = 13 \implies \varphi(N) = (p-1)^2=12^2 = 144$.  
    
    Hledáme $d$ a číslo $x\in(0,169)$ takové, že po dešifrování dostaneme hodnotu různou od $x$.

    $ $

    Nejprve si určíme $d$ a to opět za pomoci $d\cdot e \equiv 1 \pmod{\varphi (N)}$, kde $e=5$ a $\varphi(N)=144$:
    \begin{flalign*}
        5d &\equiv 1 \pmod{144} \quad // \quad 5\cdot 29=145 \equiv 1\\
        145d &\equiv 29 \pmod{144}\\
        d &\equiv 29 \pmod{144} \implies d = 29 + 144k ~(\forall k \in \Z)
    \end{flalign*}
    Nyní už stačí jen najít $x\in (0, 169)$ takové, že $x\neq \text{dec}(\text{enc}(x))$:
    \begin{flalign*}
        y &= (x^e) \mod N \quad // \quad x=2, ~ e=5, ~ N=169\\
        y &= (2^5) \mod 169\\
        y &= 32 \pmod {169}\\
        \text{---}&\text{--------------------------------}\\
        x &= (y^d) \mod N \quad // \quad y=32, ~ N = 169, ~ d =29\\
        x &= (32^{29}) \mod 169\\
        x &= 93 \pmod {169} \neq 2
    \end{flalign*}
    Takových čísel najdeme hodně, protože RSA nefunguje - kvůli špatně vypočítanému $\varphi(N)$:

    Z Eulerovy funkce $\varphi(pq)$ pro prvočísla $p\neq q$ se dá jednoduše odvodit, že $\varphi(pq)=(p-1)(q-1)$.

    Pokud ale $p=q$, tak určujeme $\varphi(p^2)$, což není $(p-1)^2$, ale $\varphi(p^2)=p(p-1)$. 

    Aby RSA fungovalo pro $p=q$, museli bychom přepočítat $\varphi(N)=13\cdot 12 = 156$ a tím pádem i přepočítat $d$.
\end{enumerate}



%%%%%%%%%%%%%%%%%%%%%%%%%%%%%%%%%%%%%%%%%%%%%%%%%%%
%%%% GUPY %%%%
%%%%%%%%%%%%%%

\section{Grupy}

\subsection{Grupy a podgrupy}

\subsubsection{.}
\subsubsection{.}

%%%%%%%%%%%%%%%%%%%%%%%%%%%%%%%%%%%%%%%%%%%%%%%%%%%%
%%%% CYKLICKE GRUPY %%%%%%
%%%%%%%%%%%%%%%%%%%%%%%%%%

\subsection{Cyklické grupy a }

\subsubsection{.}
\subsubsection{.}

\end{document}
