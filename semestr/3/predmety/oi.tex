\documentclass[10pt,a4paper]{article}

\usepackage[margin=0.7in]{geometry}
\usepackage{amssymb, amsthm, amsmath, amsfonts}
\usepackage{array, xcolor, enumitem, graphicx}
\usepackage{cancel}

\usepackage[czech]{babel}
\usepackage[utf8]{inputenc}
\usepackage[unicode]{hyperref}
\usepackage[useregional]{datetime2}

\hypersetup{
    colorlinks=true,
    linkcolor=black,
    urlcolor=blue,
    pdftitle={Ochrana informací 1},
}

% tahak - prikazy %
% \includegraphics{obrazek.png}         % import konkretniho obrazku
% a~b                                   % mezera mezi pismeny 'a' a 'b'
% a \quad b                             % velka mezera mezi pismeny 'a' a 'b'
% \sim                                  % ~
% \Longleftarrow                        % <==


% zkratky %
\newtheorem{veta}{Věta}
\newtheorem{definice}{Definice}
\newtheorem{tvrzeni}{Tvrzení}
\newtheorem{lemma}{Lemma}
\newtheorem{pozorovani}{Pozorování}
\newtheorem{dusledek}{Důsledek}
\newtheorem{priklad}{Příklad}

\newcommand{\N}{{\mathbb{N}}}       % prirozena cisla
\newcommand{\Z}{{\mathbb{Z}}}       % cela cisla
\newcommand{\Zs}{{\mathbb{Z}_n^*}}  % cela cisla vcetne nekonecna
\newcommand{\Q}{{\mathbb{Q}}}       % racionalni cisla
\newcommand{\R}{{\mathbb{R}}}       % realna cisla
\newcommand{\Cc}{{\mathbb{C}}}      % komplexni cisla
\newcommand{\F}{{\mathbb{F}}}       % teleso
\newcommand{\Pp}{{\mathcal{P}}}     % potencni mnozina

\newcommand{\hr}{{\begin{center}\par\rule{\textwidth}{0.5pt} \end{center}}}
\newcommand\makesmall{\fontsize{8pt}{11pt}\selectfont}

\renewcommand*{\contentsname}{Obsah}
\renewcommand*{\proofname}{Důkaz:}
\renewcommand*{\figurename}{\makesmall Obr.}

% hlavicka
\setlength{\parindent}{0em}

\title{Ochrana informací 1}
\date{\today}
\author{\sc Karel Velička}

\graphicspath{ {../img/} }             % obrazky ulozeny ve slozce img/

\begin{document}
\pagenumbering{arabic}
\maketitle

% podnadpis + obsah 
\begin{center}
    Beneš Antonín, RNDr., Ph.D.
\end{center}

\begin{em}
Obecně se u všeho hodí říct věci jako logování, keep-it-simple.

Dále, protože jeho ego je někde v nedosažitelných výšinách, se může hodit mu ho trošku pohladit:
\begin{itemize}
    \item Nabízí se říct např. nějaké jeho oblíbené slovo/ frázi - "Hezky česky", "ergo"
    \item Poděkovat mu předem, že předmět ve svém VOLNÉM ČASE učí a PROJEVUJE TAK MFF SLUŽBU
    \item Zeptat se ho na pár informací k 'Ochraně informací 2' - o tom on určitě milerád něco poví
\end{itemize}
\end{em}

\tableofcontents
\newpage

% obsah dokumentu

\section{Úvod}

\paragraph*{Informačním systémem (IS)} prostředek, který se používá ke správě svých informací

\paragraph*{Bezpečnostní incident} stav, kdy došlo k porušení alespoň jedné z požadovaných vlastností.

\paragraph*{Zranitelnost} nedostatek bezpečnostního systému.

\paragraph*{Dopad} finanční vyjádření incidentu

Celkově snaha minimalizovat investici, provozní náklady, očekávané ztráty

\paragraph*{Komponenty bezpečnosti} \begin{itemize} \setlength\itemsep{0em}
    \item Zákony
    \item Normy
    \item Politika
    \item Bezepčnostní cíle
    \item Kontrola prostředí
    \item Autentizace + Autorizace
    \item Separace - fyzická, časová, logická, kryptografická
    \item Integrita, dostupnost, auditabilita
\end{itemize}

\paragraph*{Možné hrozby} \begin{itemize}\setlength\itemsep{0em}
    \item Přerušení - šást ztracena/ nedosažitelná
    \item Zachycení - útočník má přístup do systému
    \item Modifikace - útočník může měnit
    \item Fabrikace - neautorizované vytvoření nového objektu
\end{itemize}

\section{Řízení bezpečnostních rizik}
\newpage

\section{Aktiva}
Jakýkoliv zdroj hodnoty (hmotný/ nehmotný)

Normy GDPR, FIPS 199, SIA (Security of Information Act) 

\subsubsection*{Klasifikace dat}
Abychom mohli pro jednotlivé kategorie navrhnout odpovídající bezpečnostní mechanismy (ty jsou založené na kontextu, obsahu, uživateli)

U klasifikace obecných aktiv je cílem „inventurní seznam“ aktiv. Pro každou třídu se stanoví minimální sada bezpečnostních opatření.


\subsubsection*{Kategorizace dat}
Seskupení typů dat na základě obdobných nároků na zabezpečení

Děli se na kategorizace dle \textit{senzitivity, kritičnosti (nepostradatelnosti), hodnoty}

\subsubsection*{Požadavky na nakládání s aktivy a informacemi}
\begin{itemize}
    \item Označování - viditelné etikety na zařízení značící důležitost, bezpečnostní úroveň, ... v záhlaví/ zápatí dokumentu
    \item Zpracování - politika, pravidla a postupy používání senzitivních dat a kritických aktiv - důležitost školení
    \item Uložení - lokace a zabezpečení uložených dat, šifrování, použití HW prostředků
    \item Deklasifikace - úorava přiřazené klasifikace - musí se dokumentovat, ideálně schválit
\end{itemize}

\subsubsection*{Role spojené se správou dat}
\begin{itemize}
    \item Vlastník - odpovědný za stanovení, jak a kým budou data používána; rozhoduje o udělení/ odebrání přístupu; plně odpovědný za data
    \item Regulátor - osoba/ agentura/ společnost určující účel a způsoby zpracování dat; odpovědný za dodržování principů, pravidel, legislativy
    \item správce - odpovědný za údržbu dat a technických prostředků zpracování
    \item Zpracovatel - odpovědný za nakládání s daty jménem vlastníka
    \item Uživatel - konzumenti dat
    \item Subjekt - ten, o kom data vypovídají
\end{itemize}

\subsubsection*{Správa dat}
\begin{itemize}
    \item Umístění - požadavky na geografické omezení zpracování a uložení dat
    \item Údržba - zpracování, analyzování a sdílení dat; řídí přístup; least priviledge prinicp
    \item Uchovávání - třeba stanovit pro každý typ dat
    \item Likvidace - zvážit schválení; třeba likvidace i nosičů (fyzicky - drcení)
    \item Remanence - smazaná data lze získat zpětně z nosičů; na cloud nelze zajistit - ukládat šifrovaně
\end{itemize}

\subsubsection*{Opatření pro zajištění bezpečnosti dat}
\begin{itemize}\setlength\itemsep{0em}
    \item Technická - firewally, filtry, šifrování
    \item Administrativní - politiky, standardy, postupy
    \item Fyzické - stráž, recepce
\end{itemize}

\newpage
\section{Architektura}

Systémy a aplikace procházejí fázemi - návrh, vývoj, testing, nasazení, údržba, vyřazení

Používají se obvykle konkrétní standardy ISO, STRIDE, PASTA

\subsection*{Obecné principy}

\subsubsection*{Separace domény}

Doména je soubor souvisejících komponent se společnými bezpečnostními atributy

Komunikace je omezená na kanály

\subsubsection*{Vrstvení (layering)}

Hierarchické strukturování systému - vyšší závisí na nižších

Je to dekompouzice $\implies$ napomáhá srozumitelnosti, atd.

\subsubsection*{Zapozdření (encapsulating)}

Objekty nezpřístupňuí data, ale metody - lepší kontrola přístupu a integrita

\paragraph*{Redundance} replikace komponent, paralelní zpracování, vyšší odolnost
\paragraph*{Virtualizace} separace; snazší zotavení

\paragraph*{Nejmenší oprávnění (least priviledge)} služby, informace dostupné na základě aktuální potřeby \textit{(need-to-know princip)}

\paragraph*{Attack surface (plocha pro útok)} souhrn všech expozic
\paragraph*{Hardening} obecně zabezpečení - vypnutí nepotřebných služeb, odstranění standardních účtů, portů, aplikací apod.

\paragraph*{Bezpečné výchozí hodnoty, havarování} po spuštění/ havarování musí mít systém bezpečné výchozí hodnoty
\paragraph*{Fail safe/open} blokuje se přístup/ zachová se dostupnost (snaha)

\subsection*{Důvěra}

\paragraph*{Keep-it-simple} jednoduchost...
\paragraph*{Důvěřuj, ale prověřuj (trust but verify)} ověřujte cokoliv, co přichází z vnějšího prostředí 
\paragraph*{Zero trust} nevěří se ničemu; všichni jsou hrozba - všechny vstupy se validují, přístupy autentizují atd.

\subsection*{Bezpečnostní modely obecně}

První fází tvorby bezpečného IS je volba vhodného bezpečnostního modelu. 
Základní požadavky bezpečnosti jsou \textit{utajení, integrita, dostupnost, anonymita}.
Předpokládejme, že umíme rozhodnout, zda danému subjektu poskytnout přístup. 
\textit{Modely poskytují pouze mechanismus pro rozhodování!}
\begin{itemize}
    \item \textbf{Jednoúrovňové modely} jsou vhodné pro případy, kdy stačí jednoduché \textit{ano/ne rozhodování},zda danému subjektu poskytnout přístup k požadovanému objektu a \textit{není nutné pracovat s klasifikací dat}.
    \item \textbf{Víceúrovňové modely} Může existovat několik stupňů senzitivity a ”oprávněnosti”.
    Tyto stupně senzitivity se dají použít k algoritmickému rozhodování o přístupu daného
    subjektu k cílovému objektu, ale také k řízení zacházení s objekty. Víceúrovňový systém
    \textit{”rozumí” senzitivitě dat} a chápe, že s nimi musí zacházet v souladu s požadavky
    kladenými na daný stupeň senzitivity. Rozhodnutí o přístupu pak nezahrnuje pouze
    prověření žadatele, ale \textit{též klasifikaci prostředí, ze kterého je přístup požadován}.
\end{itemize}

\subsection*{Modely pro speifické účely}

\subsubsection*{Chinese wall model}
Dynamický model, pravidla generována až v okamžiku používání.

Konzultant nesmí radit konkurenci, ale může radit nekonkurenci.

Objekty jedné organizace tvoří \textit{dataset}, datasety rozčleněny do \textit{tříd} (conflict of interest classes)

\textit{Sanitizovaná informace} – odstraněny ty části, které umožňují identifikovat konkrétního vlastníka

Subjekt na počátku univerzální práva (ke všem objektům)

Přístup je povolen, pokud je ve stejném datasetu/ náleží do jiné třády.

Zápis je povolen, pokud je umožněn přístup/ není čten objekt s informacemi z jiného datasetu.
\subsubsection*{Clark-Willson model}
\begin{figure}[ht]
    \includegraphics[width=.8\textwidth]{clark.png}
    \caption{Clark-Willson model - zdroj: Tahak\_Ochrany\_Informace.pdf}
    \label{normal_case}
\end{figure}

\subsubsection*{Take-Grant systém}
Systém přidělování a odebírání oprávnění; Efektivních vyhodnocování práv (v $O(n)$)

Čtyřmi základními primitiva: \textit{create, revoke, take, grant}

\subsubsection*{Military security model}
Řazení do disjunktních kategorií utajení - \textit{unclassified, confidental, secret, top secret}

Uplatnění least priviledge principu.

\subsubsection*{Svazový model (Lattice model)}
Military je případem tohoto modelu. Uplatnění relace $\leq$. Rozdělení do kategorií podle utajení.
\newpage
\subsubsection*{Graham-Denning model}
\begin{figure}[ht]
    \includegraphics[width=.65\textwidth]{graham.png}
    \caption{Graham-Denning model}
\end{figure}

\subsubsection*{Bell-LaPadula model}
\begin{figure}[ht]
    \includegraphics[width=.75\textwidth]{bell.png}
    \caption{Bell-LaPadula model - zdroj: Tahak\_Ochrany\_Informace.pdf}
\end{figure}

\subsubsection*{Biba model}
\begin{figure}[ht]
    \includegraphics[width=.75\textwidth]{biba.png}
    \caption{Biba model - zdroj: Tahak\_Ochrany\_Informace.pdf}
\end{figure}

\section{Mechanismy}
\section{Sítě}

\subsection*{Firewall}

\paragraph*{Personální firewally} filtry - analýza a odstraňuje nežádoucí objekt

\paragraph*{Aplikační firewally (proxy brány)} musí existovat specializovaná „proxy“ pro každý přenášený protokol

\paragraph*{Síťové IDS, IPS} hlídá známe vzory chování odpovídající útoku (IDS - detekuje, IPS - reaguje)

\subsection*{Protokoly}

\paragraph*{SSL}
Struktura - handshake, cipher, alert, application protocol (HTTP), record, TCP, IP

\paragraph*{IpSec} dvojice nezávislých protokolů: $\begin{cases}
    \text{AH (appliation header)} & \text{integirita packetů, HMAC; MD5}\\
    \text{ESP (encapsulated security payload)} & \text{integrita + utajení dat, HMAC} \\
    & \text{blokové šifry (AES, Blowfish)}
\end{cases}$

\subsection*{Internal Security Gateway (ISG)} \begin{itemize}
    \item Ověřuje soulad se standardy
    \item Kontroluje použití očekávaných mechanism
    \item Hlídá hazardní operace aplikací
    \item Analyzuje nebezpečný kód
\end{itemize}

\section{IAM}
\section{Provoz}
\section{Hodnocení}

$\equiv $ vyhodnocení vlastnosti objektu (pravděpodobnost, předpokládaná velikost, ...); výstupem je report s nálezy a zkoumáním

\subsection*{Návrh a validace - strategie testování}

Explicitní stanovení rozsahu a cílů zkoumání - jasně definovat, volit vhodnou metodologii (umožňuje srovnání)

\paragraph*{Audit} porovnání reálného stavu organizace s deklarovaným stavem (standardy, smluvní závazky, cíle).

\paragraph*{Standardy pro audit} Doporučuje se ISO/IEC 15408, ISO/IEC 27006, NIST 800-53A atd.

\subsubsection*{Interní audit} \begin{itemize} \setlength\itemsep{0em}
    \item Někým jiným než autorem politiky.
    \item Výhody: znalost prostředí, kultury a možnost častějšího provádění.
    \item Vhodný pro hledání slabin, rutinní kontroly patchů, ...
    \item Ideální příprava na externí audit, omezení dopadu na obchodní aktivity v případě neúspěchu auditu.
\end{itemize}

\subsubsection*{Externí audit} \begin{itemize} \setlength\itemsep{0em}
    \item Nezávislý + pohled zvenčí.
    \item Větší zkušenosti, ale menší znalost objektu
    \item Jediná možnost pro angažování dostatečně erudovaných pracovníků.
\end{itemize}
 
\subsubsection*{Audit třetí stranou} \begin{itemize} \setlength\itemsep{0em}
    \item Provádí zákazník/ jiný obchodní partner $\implies$ odpovědnost vlastníka
    \item Zaměření na provozní postupy, správa procesů, zabezpečení dat
\end{itemize}


\subsection*{Testování bezpečnostních opatření}
Slouží k identifikaci slabých míst a rizik v systému.

Kontroly logů + kódu + testů; testování zneužití, rozhraní; analýza pokrytí testů; Simulace narušení a útoku

\subsubsection*{Hodnocení slabin}
Hledat v kritických komponentách; Po nalezení provést hodnocení dopadu, relevanci a stanovit priority oprav.

\subsubsection*{Penetrační testování}

Kodex etického hackera. Snaha najít známe slabiny a zmapovat dopad + rozsah. Výsledkem je report.

Obvykle - průzkum, sken, využití poznatků, průnik, report

\subsubsection*{Syntetické transakce}
simulované aplikační transakce, kterými se ověřuje funkčnost cílových systémů a kontroluje se, že systém odpoví očekávanou odezvou

\subsection*{Shromažďování dat o bezpečnostních procesech}
Založeno na průběžném monitoringu.

Cílem je hodnocení stavu a provozu bezpečnostních opatření.

Vychází z hodnocení rizik a návrhu programu bezpečnosti - díky tomu stanoví strategie tvorby a zpracování logů.

Logování + zavede se centralizace zpracování + automatizace monitoringu.

Nasazení pokročilých metod automatického vyhodnocování (trendy, clustery, AI...)

\subsubsection*{Administrativní opatření}

Politika, pravidla; záznamy o obchůzkách, vydané ceritifikáty... Hodnotí se dopad politiky, efektivita ...

\paragraph*{Správa účtů}
Administrativní + techická opatření

\subsubsection*{Klíčové indikátory výkonu a rizika}
Je vhodné se zamýšlet nad budoucím vývojem 

\paragraph*{KPI} hodnocení stávajících opatření na základě vhodných metrik
\paragraph*{KRI} povědomí o nadcházejících hrozbách a vývoji rizik

Dobré obstarávat bezpečnostní skóre ($\#$ malware, $\#$ oprav SW, neúspěšná přihlášení apod.), a sledovat návratnost investic (vyplatilo se opatření?), 


Dále kontrola a schválení managementem; Kontrola shody (ISO, ...); Kontrola záloh; Školení a povědomí; Náprava chyb; Obnova po katastrofě
\end{document}
